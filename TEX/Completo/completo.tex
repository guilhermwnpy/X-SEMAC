\documentclass[a4paper,12pt]{report}

\usepackage[top=2cm,bottom=2cm,right=2cm,left=2cm]{geometry}
\usepackage[utf8]{inputenc}
\usepackage[T1]{fontenc}
\usepackage[portuguese]{babel}
\usepackage{xcolor,graphicx}
\usepackage[colorlinks]{hyperref}
\usepackage{listings}

\setlength\parindent{0pt}
\definecolor{blue}{RGB}{31,56,100}
\definecolor{green}{RGB}{34,139,34}


\begin{document}

\begin{titlepage}
\begin{center}


\begin{minipage}{2cm}
    \centering
    \includegraphics[height=2.5cm]{logoUFS.png}
\end{minipage}
\hfill
\begin{minipage}{11cm}
    \centering
    \includegraphics[height=2.5cm]{logoINST.png}
\end{minipage}
\hfill
\begin{minipage}{3cm}
    \centering
    \includegraphics[height=2cm]{logoDEL.png}
\end{minipage}



\vspace{2cm}

% Title and course
{\large \bfseries Material de Apoio do Curso}\\[0.5cm]
{\huge \bfseries INTERFACE HOMEM MÁQUINA MICROCONTROLADA} \\[0.5cm]

\vspace{2.5cm}

% Title of material
{\large \bfseries Material:}\\[0.5cm]
\rule{\linewidth}{0.3mm} \\[0.6cm]
{\huge \bfseries \color{blue} Interface Homem Máquina Microcontrolada} \\[0.2cm]
\rule{\linewidth}{0.3mm} \\[2cm]

% Authors and date
\noindent
\begin{minipage}{0.4\textwidth}
    \begin{flushleft} \large
        \emph{Ministrantes do Curso:}\\
        \textbf{Guilherme Franco} \\
        \textbf{Reinan Lima}
    \end{flushleft}
\end{minipage}%
\begin{minipage}{0.6\textwidth}
    \begin{flushright} \large
        \emph{Data do Curso:}\\
        \textbf{10/12/2024}
    \end{flushright}
\end{minipage}

\vfill

% Footer
{\textbf{\large Universidade Federal de Sergipe - 2024}}

\end{center}
\end{titlepage}

% \newpage

% \section*{Instalações curso IHMM}

% Para esse curso, onde iremos criar interfaces, precisamos primeiro preparar o ambiente onde vamos trabalhar, e para isso temos que instalar algumas coisas na máquina. Nosso curso vai utilizar a linguagem de programação Python para construção da interface e funções aplicadas a ela. Por isso precisamos que nosso computador seja capaz de executar os códigos em python que escrevemos, e importante também, que possamos escrever o código de forma tranquila, usando um editor de texto adequado.

% \subsection*{Instalação do Python}

% Para instalar o Python, basta ir na página oficial e ir na parte de download. Qualquer versão pode ser instalada, mas recomendamos a versão mais recente, a `3.13.0`

% \vspace{2pt}
% \href{https://www.python.org/downloads/}{PÁGINA DE DOWNLOAD}

\end{document}
